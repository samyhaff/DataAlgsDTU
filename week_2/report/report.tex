\documentclass[11pt]{article}

\usepackage[utf8]{inputenc}
\usepackage[english]{babel}
\usepackage{amsmath}
\usepackage{algorithm}
\usepackage{algpseudocode}
\usepackage[margin=2.5cm]{geometry}

\title{Algorithms and Data Structures 2}
\author{Samy Haffoudhi}
\date{\today}

\begin{document}
\maketitle

\section*{Dynamic Programming 1}
\subsection*{Manhattan Tourists}

\textbf{5.1}

We fill the table for the given example:

\begin{center}
    \begin{tabular}{ |c||c|c|c|c|c| }
        \hline
        W_{i,j} & 1 & 2 & 3 & 4 & 5 \\
        \hline
        1 & 0 & 2 & 9 & 16 & 24 \\
        \hline
        2 & 3 & 4 & 18 & 26 & 32 \\
        \hline
        3 & 12 & 10 & 24 & 35 & 41 \\
        \hline
        4 & 13 & 23 & 29 & 39 & 46 \\
        \hline
        5 & 18 & 30 & 34 & 47 & 52 \\
        \hline
    \end{tabular}
\end{center}

Thus, the maximum weight a shortest path can have is $W[5,5] = 52$.

\noindent
\textbf{5.2}

$$
W[i,j] =
\begin{cases}
0 & \text{for } $(i, j) = (1, 1)$ \\
W[i,j-1] + R[i,j-1] & \text{for } i = 1 \\
W[i-1,j] + D[i-1,j] & \text{for } j = 1 \\
\max(W[i-1, j] + D[i-1,j], W[i,j-1] + R[i,j-1]) & \text{otherwise}
\end{cases}
$$

Indeed,

\begin{itemize}
\item{$W[1,1] = 0$ since the path is empty when going from $s$ to $s$}
\item{When $i = 1$ it is only possible to walk right}
\item{When $j = 1$ it is only possible to walk down}
\item{In the last case, there were two available choices for the last
move leading to the node $\nu_{i,j}$: walking down from $\nu_{i-1,j}$
or walking right from $\nu_{i,j-1}$. We choose the one which maximizes
the sum of the weights along the path}
\end{itemize}

\noindent
\textbf{5.3}

\begin{algorithm}
\caption{Compute the maximum weigh a shortest path can have}
\label{algorithm}
\begin{algorithmic}
    \Require $n, D, R$
    \Ensure W[n,n] is the maximum weight a shortest path $s-t$ can have
    \State $W[1,1] \gets 0$
    \For{i = 2 to n}
    \State $W[i,1] \gets W[i-1,1] + D[i-1,1]$
    \EndFor
    \For{j = 2 to n}
    \State $W[1,j] \gets W[1,j-1] + R[1,j-1]$
    \EndFor
    \For{i = 2 to n}
    \For{j = 2 to n}
    \State $W[1,j] \gets \max(W[i-1, j] + D[i-1,j], W[i,j-1] + R[i,j-1])$
    \EndFor
    \EndFor
\end{algorithmic}
\end{algorithm}

We write algorithm \ref{algorithm} (page \pageref{algorithm}) based on
dynamic Programming and the recurrence from Question \textbf{5.2}.
Since we store the results to the subproblems in table $W$ of size
$ n x n $ we use $O(n^2)$ space.
The first two loops make $n - 1$ iterations each.
Regarding the nested loops, the outer one makes $n - 1$ iterations
while the inner one makes $n - 1$ iterations per iteration of the
outer loop. All other operations are done in constant time and thus the
total time of the algorithm is $O(n^2)$.

\end{document}
